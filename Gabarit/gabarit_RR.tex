\documentclass[12pt]{article}

\usepackage[T1]{fontenc}
\usepackage[latin1]{inputenc}
\usepackage[a4paper]{geometry}

\pagestyle{empty}
\geometry{top=2cm,bottom=3cm,left=2cm,right=2cm}


% No paragraph indent or paragraph skip
\parindent=0pt \parskip=0pt

\begin{document}

\centerline{\bf Instructions pour les auteurs}

\vspace{12pt}

\centerline{ {\bf A.~Auteur}$^{\rm a}$ and {\bf B.~Auteur}$^{\rm b}$}

\vspace{12pt}

\centerline{$^{\rm a}$Departement de Mathematiques}
\centerline{A-Institut}
\centerline{A-Adresse}
\centerline{author1@institut.com}

\vspace{12pt}

\centerline{$^{\rm b}$Departement de Biologie}
\centerline{B-Institut}
\centerline{B-Adresse}
\centerline{author2@institut.com}

\vspace{24pt}

{\bf Mots clefs} : Statistique, Biologie, R\'egression.

\vspace{24pt}

Le processus de s\'election est bas\'e sur un r\'esum\'e de la pr\'esentation. Chaque r\'esum\'e sera examin\'e par un relecteur. Le processus de soumission comporte trois \'etapes :
\begin{enumerate}
\item Soumission du r\'esum\'e initial avant le 8 avril 2016
\item Notification de l'acceptation du r\'esum\'e autour du 2 mai 2016
\item Soumission du r\'esum\'e final avant le 16 mai 2016
\end{enumerate}

\vspace{12pt}

La longueur du r\'esum\'e ne doit pas exc\'eder 2 pages. Le r\'esum\'e doit contenir
\begin{enumerate}
\item  le titre de la pr\'esentation,
\item  les noms des auteurs,
\item  les mots clefs,
\item  Les contacts des auteurs, incluant l'adresse postale et l'email.
\end{enumerate}

DDes modèles Word, LibreOffice, LaTeX et RMarkdown pour la pr\'eparation des 
r\'esum\'es sont disponibles sur le site web de la conf\'erence. Une fois que 
le r\'esum\'e est pr\^et, merci de le convertir en format PDF. Les autres types 
de fichiers ne seront pas accept\'es. Chaque r\'esum\'e devra \^etre soumis en 
version \'electronique via le site web de la conf\'erence.


\vspace{12pt}

%The final abstract is to be prepared using the following format:
Le r\'esum\'e est \`a pr\'eparer en respectant les consignes suivantes :

$\bullet$ Pour les auteurs Word, LibreOffice et LaTeX, marges : 2cm \`a gauche 
et \`a droite, 2cm en haut et 3cm en bas

$\bullet$ Titre en gras

$\bullet$ Nom d'auteurs en gras

$\bullet$ \`A l'exception des auteurs utilisant Rmarkdown, choisir la police 
Times New Roman font \`a 12 pt

$\bullet$ Pour les auteurs utilisant LaTeX, choisir la classe article en 12pt

$\bullet$ 1 ligne d'espace entre le titre et le nom du premier auteur

$\bullet$ 1 ligne d'espace entre les auteurs et les affiliations

$\bullet$ \`A l'exception des auteurs utilisant Rmarkdown, 2 lignes d'espace 
entre les affiliations et les mots clefs

$\bullet$ \`A l'exception des auteurs utilisant Rmarkdown, 2 lignes d'espace 
entre les mots clefs et le corps du texte

$\bullet$ \`A l'exception des auteurs utilisant Rmarkdown, 1 ligne d'espace 
entre chaque paragraphe

$\bullet$ Pas d'indentation des paragraphes

$\bullet$ \`A l'exception des auteurs utilisant Rmarkdown, 1 ligne d'espace 
entre le corps du texte et les r\'ef\'erences [1]

$\bullet$ Ne pas inclure de num\'eros de pages

$\bullet$ Inclure les figures et les \'equations

$\bullet$ ``R\'ef\'erences'' en gras comme indiqu\'e ci-dessous

$\bullet$ R\'ef\'erences num\'erot\'ees [2], pas de ligne d'espace entre les 
r\'ef\'erences.


\vspace{12pt}

\parindent=0pt
{\bf R\'ef\'erences}

[1] Liquet, B., Saracco, J. (2011). A criterion for selecting the number of slices and the dimension of the model in SIR and SAVE approaches. To appear in {\it Computational Statistics}.

[2] Liquet, B., Saracco, J. (2008). Application of the bootstrap approach to the choice of dimension and the $\alpha$ parameter in the SIR$_\alpha$ method. {\it Communications in Statistics - Simulation and Computation}, {\bf 37}(6), 1198-121



\end{document}
